\documentclass[12pt]{article}
\usepackage{graphicx}
\usepackage[none]{hyphenat}
\usepackage{graphicx}
\usepackage{listings}
\usepackage[english]{babel}
\usepackage{graphicx}
\usepackage{caption} 
\usepackage{hyperref}
\usepackage{booktabs}
\usepackage{array}
\usepackage{amsmath}   % for having text in math mode
\usepackage{extarrows} % for Row operations arrows
\usepackage{listings}
\lstset{
  frame=single,
  breaklines=true
}
  
%Following 2 lines were added to remove the blank page at the beginning
\usepackage{atbegshi}% http://ctan.org/pkg/atbegshi
\AtBeginDocument{\AtBeginShipoutNext{\AtBeginShipoutDiscard}}


%New macro definitions
\newcommand{\mydet}[1]{\ensuremath{\begin{vmatrix}#1\end{vmatrix}}}
\providecommand{\brak}[1]{\ensuremath{\left(#1\right)}}
\providecommand{\norm}[1]{\left\lVert#1\right\rVert}
\newcommand{\solution}{\noindent \textbf{Solution: }}
\newcommand{\myvec}[1]{\ensuremath{\begin{pmatrix}#1\end{pmatrix}}}
\let\vec\mathbf

\begin{document}

\begin{center}
\title{\textbf{Vector Algebra}}
\date{\vspace{-5ex}} %Not to print date automatically
\maketitle
\end{center}
\setcounter{page}{1}

\section{12$^{th}$ Maths - Chapter 10}
This is Problem-8 from Miscellaneous Exercise
\begin{enumerate}
\item Show that the points A$\myvec{1\\-2\\-8}$, B$\myvec{5\\0\\-2}$ and C$\myvec{11\\3\\7}$ are collinear, and
find the ratio in which B divides AC.\\
\solution 
To find collinearity of the given points, we use rank method. If the rank is less than 3, then the points are said to be collinear.
\begin{align}
 \myvec{1&5&11\\-2&0&3\\-8&-2&7\\1&1&1} &\xleftrightarrow[]{R_2\rightarrow R_2+2R_1} \myvec{1&5&11\\0&10&25\\-8&-2&7\\1&1&1}\\
 &\xleftarrow[]{R_3\rightarrow 8R_1 + R_3} \myvec{1&5&11\\0&10&25\\0&38&95\\1&1&1}\\
 &\xleftarrow[]{R_4\rightarrow -R_1 + R_4} \myvec{1&5&11\\0&10&25\\0&38&95\\0&-4&-10}\\
 &\xleftarrow[]{R_3\rightarrow -\frac{-38R_2}{10} + R_3} \myvec{1&5&11\\0&10&25\\0&0&0\\0&-4&-10}\\
 &\xleftarrow[]{R_4\rightarrow \frac{4R_2}{10} + R_4} \myvec{1&5&11\\0&10&25\\0&0&0\\0&0&0}
\end{align}
Therefore, the rank of the matrix is 2. Hence, the points are collinear as the rank of the matrix is less than 3.\\ \\

Let us assume $B$ divides $AC$ in $k:1$ ratio. Therefore, using section formula we get:
\begin{align}
    \implies \Vec{B} &= \frac{k\Vec{C}+\Vec{A}}{k+1}
    \label{eq:5}
\end{align}
Substituting the values of $\vec{A},\vec{B}$ and $\vec{C}$ in \eqref{eq:5}
\begin{align}
    \myvec{5 \\ 0 \\ -2} &= \frac{\myvec{11 \\ 3 \\ 7}k + \myvec{1 \\ -2 \\ -8}}{k+1}\\
    \label{eq:6}
    \myvec{5 \\ 0 \\ -2} &=\frac{1}{k+1}\myvec{1+11k\\-2+3k\\-8+7k}
    \label{eq:7}
\end{align}
Simplifying \eqref{eq:7} yields,
\begin{align}
    0 &=\frac{-2+3k}{1+k}\\
\implies          k &=\frac{2}{3}
\end{align}
Also,
\begin{align}
          5 &=\frac{1+11k}{1+k}\\
    \implies      k &=\frac{2}{3}
\end{align}
Also,
\begin{align}
          -2 &=\frac{-8+7k}{1+k}\\
    \implies      k &=\frac{2}{3}
\end{align}
Hence the desired ratio is $\dfrac{2}{3}$. 
\end{enumerate}

\end{document}

