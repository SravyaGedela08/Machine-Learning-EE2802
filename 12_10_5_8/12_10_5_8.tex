\documentclass[12pt]{article}
\usepackage{graphicx}
\usepackage[none]{hyphenat}
\usepackage{graphicx}
\usepackage{listings}
\usepackage[english]{babel}
\usepackage{graphicx}
\usepackage{caption} 
\usepackage{hyperref}
\usepackage{booktabs}
\usepackage{array}
\usepackage{amsmath}   % for having text in math mode
\usepackage{extarrows} % for Row operations arrows
\usepackage{listings}
\lstset{
  frame=single,
  breaklines=true
}
  
%Following 2 lines were added to remove the blank page at the beginning
\usepackage{atbegshi}% http://ctan.org/pkg/atbegshi
\AtBeginDocument{\AtBeginShipoutNext{\AtBeginShipoutDiscard}}


%New macro definitions
\newcommand{\mydet}[1]{\ensuremath{\begin{vmatrix}#1\end{vmatrix}}}
\providecommand{\brak}[1]{\ensuremath{\left(#1\right)}}
\providecommand{\norm}[1]{\left\lVert#1\right\rVert}
\newcommand{\solution}{\noindent \textbf{Solution: }}
\newcommand{\myvec}[1]{\ensuremath{\begin{pmatrix}#1\end{pmatrix}}}
\let\vec\mathbf

\begin{document}

\begin{center}
\title{\textbf{Vector Algebra}}
\date{\vspace{-5ex}} %Not to print date automatically
\maketitle
\end{center}
\setcounter{page}{1}

\section{12$^{th}$ Maths - Chapter 10}
This is Problem-8 from Miscellaneous Exercise
\begin{enumerate}
\item Show that the points A$(1, – 2, – 8)$, B$(5, 0, – 2)$ and C$(11, 3, 7)$ are collinear, and
find the ratio in which B divides AC.\\
\solution 
 We know that points $\vec{A}, \vec{B} \text{ and } \vec{C}$ are collinear, if
\begin{align}
  \label{eq:1}
\text{rank}\myvec{ 
	\vec{AB} \\ \vec{AC}
}    &= \text{ } 1 \\   
\myvec{ 
\vec{AB} \\
\vec{AC}
}    &=   		\myvec{
        		4 & 2 & 6\\
        		10 & 5 & 15 \\
}
\end{align}
Performing a sequence of row reduction operations \\
\begin{align}
\xleftrightarrow[{R_1\rightarrow \frac{10}{4}R_1}]{{R_2\rightarrow R_2-R_1}}  \myvec{
  10 & 5 & 15 \\
  0 & 0 & 0 \\
}    \\
\xleftrightarrow[]{{R_1\rightarrow \frac{4}{10}R_1}}  \myvec{
  4 & 2 & 6\\
  0 & 0 & 0\\
}
\end{align}
Therefore, the rank of the matrix is 1. Hence, referring to equation \ref{eq:1}, the points are collinear as the rank of the matrix is equal to 1.\\ \\

Let us assume B divides AC in k:1 ratio. Therefore, we get:
\begin{align}
    \implies B &= \frac{kC+A}{k+1}\\
    \label{eq:5}
    &= \frac{(11k,3k,7k)+(1,-2,-8)}{k+1}
    \label{eq:6}
\end{align}
Substituting B = $(5,0,2)$
\begin{align}
    \implies \frac{(11k,3k,7k)+(1,-2,-8)}{k+1} &= (5,0,-2)\\
    \label{eq:7}
     \implies (11k+1,3k-2,7k-8) &= (5k+5,0,-2k-2)\\
    \label{eq:8}
    \implies k &= \frac{2}{3}
    \label{eq:9}
\end{align}

B divides AC in 2:3 ratio.

\end{enumerate}

\end{document}

